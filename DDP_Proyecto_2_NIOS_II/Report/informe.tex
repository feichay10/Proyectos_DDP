\documentclass[11pt]{report}

% Paquetes y configuraciones adicionales
\usepackage{graphicx}
\usepackage[export]{adjustbox}
\usepackage{caption}
\usepackage{float}
\usepackage{titlesec}
\usepackage{geometry}
\usepackage[hidelinks]{hyperref}
\usepackage{titling}
\usepackage{titlesec}
\usepackage{parskip}
\usepackage{wasysym}
\usepackage{tikzsymbols}
\usepackage{fancyvrb}
\usepackage{xurl}
\usepackage{hyperref}
\usepackage{subcaption}
\usepackage{amsmath}
\usepackage{listings}
\usepackage{xcolor}

\usepackage[spanish]{babel}

\newcommand{\subtitle}[1]{
  \posttitle{
    \par\end{center}
    \begin{center}\large#1\end{center}
    \vskip0.5em}
}

% Configura los márgenes
\geometry{
  left=2cm,   % Ajusta este valor al margen izquierdo deseado
  right=2cm,  % Ajusta este valor al margen derecho deseado
  top=3cm,
  bottom=3cm,
}

% Configuración de los títulos de las secciones
\titlespacing{\section}{0pt}{\parskip}{\parskip}
\titlespacing{\subsection}{0pt}{\parskip}{\parskip}
\titlespacing{\subsubsection}{0pt}{\parskip}{\parskip}

% Redefinir el formato de los capítulos y añadir un punto después del número
\makeatletter
\renewcommand{\@makechapterhead}[1]{%
  \vspace*{0\p@} % Ajusta este valor para el espaciado deseado antes del título del capítulo
  {\parindent \z@ \raggedright \normalfont
    \ifnum \c@secnumdepth >\m@ne
        \huge\bfseries \thechapter.\ % Añade un punto después del número
    \fi
    \interlinepenalty\@M
    #1\par\nobreak
    \vspace{10pt} % Ajusta este valor para el espacio deseado después del título del capítulo
  }}
\makeatother

% Configura para que cada \chapter no comience en una pagina nueva
\makeatletter
\renewcommand\chapter{\@startsection{chapter}{0}{\z@}%
    {-3.5ex \@plus -1ex \@minus -.2ex}%
    {2.3ex \@plus.2ex}%
    {\normalfont\Large\bfseries}}
\makeatother

% Configurar los colores para el código
\definecolor{codegreen}{rgb}{0,0.6,0}
\definecolor{codegray}{rgb}{0.5,0.5,0.5}
\definecolor{codepurple}{rgb}{0.58,0,0.82}
\definecolor{backcolour}{rgb}{0.95,0.95,0.92}

% Configurar el estilo para el código
\lstdefinestyle{mystyle}{
  backgroundcolor=\color{backcolour},   
  commentstyle=\color{codegreen},
  keywordstyle=\color{magenta},
  numberstyle=\tiny\color{codegray},
  stringstyle=\color{codepurple},
  basicstyle=\ttfamily\footnotesize,
  breakatwhitespace=false,         
  breaklines=true,                 
  captionpos=b,                    
  keepspaces=true,                 
  numbers=left,                    
  numbersep=5pt,                  
  showspaces=false,                
  showstringspaces=false,
  showtabs=false,                  
  tabsize=2
}

%==============================================================================
% Cosas para la documentación LateX
% % Sangría
% \setlength{\parindent}{1em}Texto

% % Quitar sangría
% \noindent

% % Punto
% \CIRCLE \ \ \textbf{Texto} \emph{algo}
% \begin{itemize}
%   \item \textbf{Negrita:} Texto
%   \item \textbf{Negrita:} Texto
% \end{itemize}

% % Introducir código
% \begin{center}
%   \begin{BVerbatim}
%     ... Código
%   \end{BVerbatim}
% \end{center}

% Poner una imagen
% \begin{figure}[H]
%   \centering
%   \includegraphics[scale=0.55]{img/}
%   \caption{Exportación de la base de datos en formato sql}
%   \label{fig:exportación de la base de datos en formato sql}
% \end{figure}

% Poner dos imágenes
% \begin{figure}[H]
%   \begin{subfigure}{0.5\textwidth}
%     \centering
%     \includegraphics[scale=0.45]{img/}
%     \caption{Texto imagen 1}
%   \end{subfigure}%
%   \begin{subfigure}{0.5\textwidth}
%     \centering
%     \includegraphics[scale=0.45]{img/}
%     \caption{Texto imagen 2}
%   \end{subfigure}
%   \caption{Texto general}
% \end{figure}

% % Poner una tabla
% \begin{table}[H]
%   \centering
%   \begin{tabular}{|c|c|c|c|}
%     \hline
%     \textbf{Campo 1} & \textbf{Campo 2} & \textbf{Campo 3} & \textbf{Campo 4} \\ \hline
%     Texto & Texto & Texto & Texto \\ \hline
%     Texto & Texto & Texto & Texto \\ \hline
%     Texto & Texto & Texto & Texto \\ \hline
%     Texto & Texto & Texto & Texto \\ \hline
%   \end{tabular}
%   \caption{Nombre de la tabla}
%   \label{tab:nombre de la tabla}
% \end{table}

% % Poner codigo de un lenguaje a partir de un archivo
% \lstset{style=mystyle}
% The next code will be directly imported from a file
% \lstinputlisting[language=Python]{code.py}

% “Texto entre comillas dobles”

%==============================================================================

\begin{document}

% Portada del informe
% Portada del informe
\title{Estudio de la influencia de la arquitectura y de la organización interna en el rendimiento}
\subtitle{Diseño de Procesadores}
\author{Carlos Pérez Fino \texttt{alu0101340333@ull.edu.es} \and Cheuk Kelly Ng Pante \texttt{alu0101364544@ull.edu.es}}
\date{\today}

\maketitle

\pagestyle{empty} % Desactiva la numeración de página para el índice

% Índice
\tableofcontents

% Nueva página
\cleardoublepage

\pagestyle{plain} % Vuelve a activar la numeración de página
\setcounter{page}{1} % Reinicia el contador de página a 1

% Secciones del informe
% Capitulo 1
\chapter{Introducción}
El objetivo de este proyecto es abordar la implementación hardware de un procesador, en este caso el Nios II, que optimice el rendimiento
en la resolución de un problema concreto, en este caso es la de aplicar un filtro de convolución sobre una imagen original.

Para empezar el proyecto se ha hecho una actividad previa que fue la del tutorial sobre el desarrollo de Nios II y una vez finalizado el 
tutorial se ha procedido a la realización de algunos programas de prueba para familiarizarse con el entorno de desarrollo y con el
procesador Nios II. Tras probar que el entorno de desarrollo y el procesador funcionan correctamente se ha procedido a mejorar el diseño
midiendo el rendimiento de cada mejora. Se llegó a la conclusión de que al introducir diferentes tipos de memoria este tardaba casi lo mismo
para las diferentes mejoras. Para comprobar los tiempos se realizó el siguiente programa:
\lstset{style=mystyle}
\lstinputlisting[language=C]{./code/test_memories.c}

% Capitulo 2
\chapter{Desarrollo del filtro de convolución, filtro Gaussiano}
Para el desarrollo del filtro de convolución se ha utilizado el filtro Gaussiano. El filtro Gaussiano es un filtro que se utiliza para
suavizar una imagen. Se ha utilizado este filtro porque es un filtro que se utiliza mucho en el procesamiento de imágenes y es un filtro
que se puede aplicar a cualquier imagen. El filtro Gaussiano se aplica a una imagen para suavizarla y eliminar el ruido de la imagen.

\section{Implementación del filtro Gaussiano}
Para la implementación del filtro Gaussiano se ha utilizado el siguiente código:
\lstset{style=mystyle}
\lstinputlisting[language=C]{./code/gaussian_filter.c}

Este algoritmo se ha implementado en C y lo primero lo que se hace es definir el kernel del filtro Gaussiano. El kernel del filtro
Gaussiano es una matriz de 3x3 que se utiliza para aplicar el filtro a la imagen. Tiene la siguiente forma:

\begin{center}
$\begin{bmatrix}
  1 & 2 & 1 \\
  2 & 4 & 2 \\
  1 & 2 & 1 \\
\end{bmatrix}$
\end{center}

Este kernel se utiliza para aplicar el filtro Gaussiano a la imagen. Para aplicar el filtro Gaussiano a la imagen se recorre la imagen
y se aplica el filtro a cada píxel de la imagen. Para aplicar el filtro a un píxel se recorre la matriz del kernel y se multiplica cada
elemento de la matriz por el píxel de la imagen correspondiente. Una vez se han multiplicado todos los elementos de la matriz del kernel
por el píxel de la imagen se suman todos los resultados y se divide entre 16. El resultado de esta operación es el nuevo valor del píxel
de la imagen que se está procesando.


\end{document}